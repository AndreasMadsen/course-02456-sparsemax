
\documentclass[a4paper]{article}

\usepackage[utf8]{inputenc}	% Flere sprog tegnsæt (fx æøå)
\usepackage[english]{babel}	% Engelsk orddeling og caption tekst
\usepackage[T1]{fontenc}		% Brug 8-bit front
\usepackage{lmodern}		% Vektor front
\usepackage{amsmath} % pænere OR code
\usepackage{graphicx}	% Kompatibilitet til visning af pixel billeder (.png, .jpg, .gif)
\usepackage{epstopdf}	% Kompatibilitet til visning af vector billeder (.eps)
\usepackage{float}		% TIllader H som positions parameter
%\usepackage{mathtools}	% Det meste matematik (indeholder ams­math og rettelser)
\usepackage{amssymb}	% Flere matematiske symboler
\usepackage{xfrac}		% Flere fracs (\sfrac{}{})
\usepackage{qtree}		% Tableau træ
\usepackage{listings}	% Indsæt code
\usepackage{fancyhdr}	% Side hoved og sidefod
\usepackage{todonotes}	% Cool todo notes, [disable] skjuler todos
\usepackage{parskip}	% Tillader paragraph vertical margin
\usepackage{url}		% Tillader \url formatering
\usepackage{subcaption}	% Tilader subfigure og subtable samt captions i dem
\usepackage{sectsty}
\usepackage{nth}
\usepackage{csquotes}	% Anbefalet package for BibLaTeX
\usepackage[backend=biber,style=ieee]{biblatex}				% Benyt BibLaTeX til formatering
\usepackage{tabularx}
\usepackage {tikz} %drawing graphs
\usetikzlibrary{positioning}

\usepackage[bookmarks,bookmarksnumbered,hidelinks]{hyperref} % clickable pdf (til sidst)



%listing settings, æøå support, font config, line number, left lines
\lstset{
    breakatwhitespace=false, breaklines=true,
    inputencoding=utf8, extendedchars=true,
 keywordstyle=\color{blue}\ttfamily,
    literate={å}{{\aa}}1 {æ}{{\ae}}1 {ø}{{\o}}1 {Å}{{\AA}}1 {Æ}{{\AE}}1 {Ø}{{\O}}1,
    keepspaces=true, basicstyle=\small\ttfamily,
    showstringspaces=false,
    frame=L, numbers=left, numberstyle=\scriptsize\color{gray},
} 

% Referencer bliver i to trin, #section.#count
\numberwithin{equation}{section}
\numberwithin{figure}{section}
\numberwithin{table}{section}
\addbibresource{sources.bib}					% Tilføjer sources.bib som reference katalog
\setcounter{secnumdepth}{5}					% Tæl paragraph sektioner
\subsubsectionfont{\fontsize{11}{8}\selectfont}		% Gør subsubsection lidt større end paragraph
\setlength{\marginparwidth}{80pt} 				% Mere brede på margin notes og todos
\setlength{\parindent}{15pt}					% Giver lidt luft imellem afsnitene
\setlength{\parindent}{0cm}   					% Deaktiver afsnit indrykning
\DeclareGraphicsExtensions{.pdf,.eps,.png,.jpg,.gif}	% ændre til .png, .jpg for hurtig visning
\DeclareMathOperator*{\argmin}{arg\,min}
\pagestyle{fancy}
\fancyhead[L]{}
\fancyhead[R]{}

\title{	
\normalsize{DTU - course 02456: Fall 2016}\\
\Large Project: Implementing Sparsemax for RNNs using TensorFlow
}

\author
{ \normalsize
	Andreas Madsen(s123598) - s123598@student.dtu.dk\\
	Marco Dal Farra Kristensen(s152630) - s152630@student.dtu.dk\\
	Frederik Wolgast Rørbech(s123956) - s123956@student.dtu.dk
} % Your name
\date{} % Today's date or a custom date

\begin{document}
\maketitle
\section*{Synopsis}
\subsection*{Introduction}
This project aims at implementing the Sparsemax activation function \footnote{"From Softmax to Sparsemax:
	A Sparse Model of Attention and Multi-Label Classification"} in TensorFlow. The Sparsemax transformation is similar to Softmax but instead of dense probability distribution gives a sparse distribution. This could be useful in many cases but this report will focus on the application for RNNs where sparse attention could be particularly useful (for instance when encoding-decoding long sentences).

\subsection*{Datasets}
The focus of this report isn't on achieving a certain result on a specific dataset. The TensorFlow implementation will have to be tested on a variety of benchmark datasets with "standard" network structures. As such the datasets has not been chosen yet.

\subsection*{Methods}
The TensorFlow implementation will be tested using a "RNN Encoder-Decoder" model. Such an RNN benefits from an "attention" mechanism. It is attention that will be made sparse using Sparsemax.

\subsection*{Expected Results}
It's expected that Sparsemax will give better performance than softmax in the settings mentioned above. The implementation is expected to be slightly slower than softmax.



\end{document}
